\documentclass[a4paper]{book}
\usepackage[times,inconsolata,hyper]{Rd}
\usepackage{makeidx}
\usepackage[utf8,latin1]{inputenc}
% \usepackage{graphicx} % @USE GRAPHICX@
\makeindex{}
\begin{document}
\chapter*{}
\begin{center}
{\textbf{\huge sprof}}
\par\bigskip{\large \today}
\end{center}
\begin{description}
\raggedright{}
\item[Type]\AsIs{Package}
\item[Title]\AsIs{Profiling, timing and optimization utilitites}
\item[Version]\AsIs{0.0-1}
\item[Date]\AsIs{2013-05-26}
\item[Author]\AsIs{G. Sawitzki}
\item[Maintainer]\AsIs{G. Sawitzki }\email{gsawitzki@r-forge.r-project.org}\AsIs{}
\item[Description]\AsIs{Support utilitites}
\item[License]\AsIs{GPL-3}
\end{description}
\Rdcontents{\R{} topics documented:}
\inputencoding{utf8}
\HeaderA{sprof-package}{sprof: R profiling}{sprof.Rdash.package}
\aliasA{sprof}{sprof-package}{sprof}
\keyword{utilities}{sprof-package}
%
\begin{Description}\relax
Profiling, timing and optimization utilitites
\end{Description}
%
\begin{Details}\relax

\Tabular{ll}{
Package: & sprof\\{}
Type: & Package\\{}
Version: & 0.0-1\\{}
Date: & 2013-05-26\\{}
License: & GPL-2\\{}
}
\code{readProf} reads a profile file and returns a composite structure.
\end{Details}
%
\begin{Author}\relax
G<c3><bc>nther Sawitzki <gsawitzki@r-forge.r-project.org>
\end{Author}
%
\begin{SeeAlso}\relax
\code{\LinkA{proftools}{proftools}}.

\code{\LinkA{profr}{profr}}.

\code{\LinkA{profr}{profr}}.

\end{SeeAlso}
\inputencoding{utf8}
\HeaderA{readProf}{Read \code{\LinkA{Rprof}{Rprof}} Output files and Stack Logs}{readProf}
%
\begin{Description}\relax
Read a log of stack entries, such as the output of the \code{\LinkA{Rprof}{Rprof}} function, and generate a more accessible representation.
\end{Description}
%
\begin{Usage}
\begin{verbatim}
readProf(filename = "Rprof.out", chunksize = 5000,
	interval = 0.02, head=c("auto"))
\end{verbatim}
\end{Usage}
%
\begin{Arguments}
\begin{ldescription}
\item[\code{filename}] Name of a file produced by \code{Rprof()}.
\item[\code{chunksize}] Number of lines to read at a time.
\item[\code{interval}] real: time interval between samples, in s.
\item[\code{head}] \code{c("auto", "none", "Rprofmem")} to interpret control information as provided by \code{Rprof} or \code{Rprofmem}. See details.
\end{ldescription}
\end{Arguments}
%
\begin{Details}\relax
This function reads a log file of stacks, one stack snapshot per line, stack entries separated by space.

As profiling output file could be very large, it
is read in blocks of \code{chunksize} lines.  Increasing \code{chunksize}
will make the function run faster if sufficient memory is available.

The input format is controlled by the \code{head} argument. Format \code{"auto"} tries to detect control lines as interspersed by \code{Rprof}. These lines are not included in the output.

\code{"none"} igores all control information and includes these lines as strange stacks.  

\code{"Rprofmem"} isolates headers as provided by  \code{Rprofmem}. \code{new page} entries are encoded as \code{malloc} requests with length 0.
\end{Details}
%
\begin{Value}
A list with components
\begin{ldescription}
\item[\code{firstline}] A verbatim copy of the first line of the input file. 
Typically this contains timing or formatting information.
\item[\code{nodes}] A vector of node names. 
This may include stray entries from interperspersed lines.
\item[\code{stacks}] A vector of unique stacks found in input, stored as verbatim copies.
\item[\code{stacksrenc}] A list of unique stacks in top down order (top first), 
encoded as vectors or references to \code{stacks}.
\item[\code{data}] A vector encoding the data file as referenes to  \code{stacks}.
\item[\code{timesRLE}] Vector of sampling intervals, in miliseconds. Run-length encoded.
\item[\code{freq}] A frequency table summarizing \code{data}.

\end{ldescription}
\end{Value}
%
\begin{Author}\relax
Günther Sawitzki <gsawitzki@r-forge.r-project.org>, based on the code of \code{\LinkA{summaryRprof}{summaryRprof}}
\end{Author}
%
\begin{References}\relax
  \url{http://sintro.r-forge.r-project.org/} 
\end{References}
%
\begin{SeeAlso}\relax
\code{\LinkA{summaryRprof}{summaryRprof}}

\code{\LinkA{summaryRprof}{summaryRprof}}

\code{\{\LinkA{readRprofileData}{readRprofileData}\} and  \code{\{flatProfile\}} in \code{library(proftools)}.


\code{\{parse\_rprof\}} in \code{library(profr)}.


  The chapter on ``Tidying and profiling R code'' in
  ``Writing \R{} Extensions'' (see the \file{doc/manual} subdirectory
  of the \R{} source tree).

  

  
\code{\LinkA{Rprof}{Rprof}} is a sampling profiler.

  \code{\LinkA{tracemem}{tracemem}} traces copying of an object via the C function
  \code{duplicate}.

  \code{\LinkA{Rprofmem}{Rprofmem}} is a non-sampling memory-use profiler.

  \url{http://developer.r-project.org/memory-profiling.html}
}

## Not run: 
\#\# Rprof() is not available on all platforms
profinterval <- 0.001
simruns <- 100

n <- 10000
x <- runif(n)
y0 <- 2+ 3 * x

sinknull <- textConnection(NULL, "w"); sink(sinknull)
Rprof(tmp <- tempfile(), interval = profinterval)
for (i in 1:simruns) \{y <- y0 +  rnorm(n); xxx<- summary(lm(y\textasciitilde{}x))\}
Rprof()

Rprof\_out <- readProf(tmp)

unlink(tmp)
sink(); close(sinknull)

str(Rprof\_out)

## End(Not run)


utilities
\end{SeeAlso}
\inputencoding{utf8}
\HeaderA{summary\_prof}{Summary for profiles}{summary.Rul.prof}
\keyword{utilities}{summary\_prof}
%
\begin{Description}\relax
Print a summary for the output of \code{\LinkA{readProf}{readProf}}.
\end{Description}
%
\begin{Usage}
\begin{verbatim}
summary_prof(x)
\end{verbatim}
\end{Usage}
%
\begin{Arguments}
\begin{ldescription}
\item[\code{x}] A data structure as returned by  \code{\LinkA{readProf}{readProf}}.

\end{ldescription}
\end{Arguments}
%
\begin{Value}
None.
\end{Value}
%
\begin{References}\relax
  \url{http://sintro.r-forge.r-project.org/} 
\end{References}
%
\begin{SeeAlso}\relax
\code{\LinkA{summaryRprof}{summaryRprof}}
\end{SeeAlso}
%
\begin{Examples}
\begin{ExampleCode}
	## Not run: 
## Rprof() is not available on all platforms
profinterval <- 0.001
simruns <- 100

n <- 10000
x <- runif(n)
y0 <- 2+ 3 * x

sinknull <- textConnection(NULL, "w"); sink(sinknull)
Rprof(tmp <- tempfile(), interval = profinterval)
for (i in 1:simruns) {y <- y0 +  rnorm(n); xxx<- summary(lm(y~x))}
Rprof()

Rprof_out <- readProf(tmp)

unlink(tmp)
sink(); close(sinknull)

summary_prof(Rprof_out)

## End(Not run)

\end{ExampleCode}
\end{Examples}
\printindex{}
\end{document}
